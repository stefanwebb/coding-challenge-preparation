\documentclass[12pt]{article}
\usepackage[utf8]{inputenc} % allow utf-8 input
\usepackage{amssymb,amstext,amsmath, amsthm}
\usepackage{wrapfig}
\usepackage{graphicx}
\usepackage{hyperref}
\usepackage{soul,color}
%\usepackage[left=1in, right=1in, top=1in, bottom=1in]{geometry}
\usepackage{listings}

% if you need to pass options to natbib, use, e.g.:
% \PassOptionsToPackage{numbers, compress}{natbib}
% before loading nips_2018

% ready for submission
% \usepackage{nips_2018}

% to compile a preprint version, e.g., for submission to arXiv, add
% add the [preprint] option:
%\usepackage[preprint]{nips_2018}
\usepackage[preprint]{nips_2018}

% to compile a camera-ready version, add the [final] option, e.g.:
% \usepackage[final]{nips_2018}

% to avoid loading the natbib package, add option nonatbib:
% \usepackage[nonatbib]{nips_2018}

\usepackage[utf8]{inputenc} % allow utf-8 input
\usepackage[T1]{fontenc}    % use 8-bit T1 fonts
\usepackage{hyperref}       % hyperlinks
\usepackage{url}            % simple URL typesetting
\usepackage{booktabs}       % professional-quality tables
\usepackage{amsfonts}       % blackboard math symbols
\usepackage{nicefrac}       % compact symbols for 1/2, etc.
\usepackage{microtype}      % microtypography

% packages from ICML template
\usepackage{graphicx}
\usepackage{times}
\newcommand{\theHalgorithm}{\arabic{algorithm}}

\usepackage{amssymb,amstext,amsmath, amsthm}

\usepackage{paralist}
\usepackage{caption}
\usepackage{subcaption}
\usepackage{setspace}
\usepackage{multicol}
%\usepackage{subfig}
\usepackage{adjustbox}
\usepackage{array}
\usepackage{multicol}
\usepackage{bbm}
\usepackage{setspace}
\usepackage{todonotes}
\usepackage{wrapfig}

\long\def\remark#1{% for notes in the margin -- from Norman Ramsey
    \ifvmode\else
        \unskip\raisebox{-3.5pt}[0pt][0pt]{\rlap{$\scriptstyle\diamond$}}%
    \fi
    \setlength\marginparwidth{1.5cm}
    \marginpar{\raggedright\hbadness=10000
    \parindent=8pt \parskip=2pt
    \def\baselinestretch{0.8}\tiny
    \itshape\noindent #1\par}}

\newcolumntype{V}{>{\centering\arraybackslash} m{.4\linewidth} }

%\input{preamble}

%Naughtyness for space
\setlength{\parskip}{0.3em}
\usepackage{titlesec}
\titlespacing\section{0pt}{4pt plus 2pt minus 2pt}{-2pt plus 2pt minus 0pt}
\titlespacing\subsection{0pt}{4pt plus 2pt minus 2pt}{-2pt plus 2pt minus 0pt}
\titlespacing\subsubsection{0pt}{4pt plus 2pt minus 2pt}{-2pt plus 2pt minus 0pt}

\newtheorem{theorem}{Theorem}
\newtheorem{lemma}{Lemma}

%\linespread{2.0}
\allowdisplaybreaks
%\doublespace


\title{Coding Challenge Notes}

% The \author macro works with any number of authors. There are two
% commands used to separate the names and addresses of multiple
% authors: \And and \AND.
%
% Using \And between authors leaves it to LaTeX to determine where to
% break the lines. Using \AND forces a line break at that point. So,
% if LaTeX puts 3 of 4 authors names on the first line, and the last
% on the second line, try using \AND instead of \And before the third
% author name.

\author{
  Stefan Webb \\
  % Department of Engineering Science \\
  University of Oxford \\
}


\begin{document}
\lstset{language=Python,xleftmargin=12pt,basicstyle=\ttfamily\footnotesize,belowskip=0pt,tabsize=2,breaklines=true}
\newcommand{\python}[1]{\lstinline[columns=fixed]{#1}}

\suppressfloats
% \nipsfinalcopy is no longer used

\setlength{\abovedisplayskip}{3.5pt}
\setlength{\belowdisplayskip}{3.5pt}
\setlength{\abovedisplayshortskip}{3.5pt}
\setlength{\belowdisplayshortskip}{3.5pt}	

\maketitle

%\begin{abstract}
%  \input{abstract}
%\end{abstract}

\section{Unorganized}

\section{Useful functions}
\begin{itemize}
  \item \python{reversed(s)} reverses an iterable and returns an iterator.
\end{itemize}

\section{Converting types}
\subsection{How do I convert a list of chars/strings to string?}
\begin{lstlisting}[numbers=left]
>>> string = ['a', 'b', 'c']
>>> "".join(strings)
\end{lstlisting}

\subsection{How do I convert a number to a string/vice versa?}
\begin{lstlisting}[numbers=left]
>>> a = str(6.24)
>>> b = int("8")
>>> c = float("1.25")
\end{lstlisting}

\section{String manipulation}
\subsection{How do I convert from a character to its ASCII/vice versa?}
\begin{lstlisting}[numbers=left]
>>> n = ord('a')
>>> a = chr(n)
\end{lstlisting}

\subsection{How to I convert a character/string to lower/upper case?}
\begin{lstlisting}[numbers=left]
>>> 'Abc'.lower()
>>> 'Abc'.upper()
\end{lstlisting}

\section{Discriminating between different types of characters}
\subsection{How do I determine if a char is punctuation?}
\begin{lstlisting}[numbers=left]
>>> import string
>>> if c is in string.punctuation:
>>>   ...
\end{lstlisting}

\subsection{How do I determine if a char is whitespace?}
\begin{lstlisting}[numbers=left]
>>> if c.isspace():
>>>   ...
\end{lstlisting}

\subsection{How do I determine if a char is alphabetical/alphanumeric?}
\begin{lstlisting}[numbers=left]
>>> if c.isalpha():
>>>   ...
\end{lstlisting}
\begin{lstlisting}[numbers=left]
>>> if c.isalnum():
>>>   ...
\end{lstlisting}

\section{Common gotchas}
\begin{itemize}
\item \python{range(n)} goes from $0$ to $n-1$, \emph{not} up to $n$.
\item There is neither \python{++} nor \python{--} operators in Python. Use \python{+= 1} and \python{-= 1} instead.
\end{itemize}


%\bibliography{dphil}
%\bibliographystyle{icml2018}

\end{document}

%%% Local Variables:
%%% mode: latex
%%% TeX-master: t
%%% End:
